%! TEX root = ../main.tex
\documentclass[main]{subfiles}

\begin{document}



\section{開発進捗と今後}
 今学期は,DBを元に権限付きGraphQL APIを作成,認証フローを構築,Web上でNode.jsを動作させるNext.jsアプリケーションのフロントエンド機能,認証・技術スタック選択・実行ログ確認画面のUIを実装した.\\
 技術スタック選択機能について,単一のライブラリインストールには他のライブラリが要求されることがほとんどである.しかし,「他のライブラリ」をフェッチするAPI機能は提供されておらず,npmライブラリ作成時の必要条件でもないため,機械的に読み取ることが困難であった.\\
 今回は,豊富な技術の選択肢を提供することよりも,確実な環境構築を促すことが優先だと考える.従って,現状では,StackoverflowやGitHub Topicsのランキングを元にした注目度・知名度の高いライブラリを抽出し,事前に全てのインストールコマンドを本システム側dbに揃えておくことで対応するという方法が,技術の流行をある程度担保しつつ確実な環境構築を促すのに最適であると考え,実装を進めている.
\end{document}
