%! TEX root = ../main.tex
\documentclass[main]{subfiles}

\begin{document}


%%
%%
%% section や subsection を必要に応じて用い,本文を記述する.
\section{はじめに }
本研究では,モダンWebアプリケーションの環境構築,特に複数人で行うWebフロントエンド開発の初期段階における環境構築の課題について考察し,
改善を図るためのWebアプリケーション,「Teamenver」の開発・検証を行う.
近年のWebフロントエンド界隈では,ライブラリやツールが頻繁に登場し,開発の際に適切に選択できないという問題が生じている.
また,選定後もライブラリ選定の意図が共有されないため,チームメンバーの納得を得られているかが不透明である.
また,更なる提案やチームでのコミュニケーションを阻害する.Teamenverを用いた開発により,初学者メンバーを含むチームでも,
適切な技術選択が行われ,かつ再利用可能となる.また,選定の意図が文書化されて保存されることで,メンバーの納得を得られた開発が可能となり,
選定後の改善にも役立つ.Teamenverの有効性の検証として,実際の開発を通して,Teamenverありとなしの開発チームでどのように開発体験に
差が出たのかを調査する.
\end{document}
