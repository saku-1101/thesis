%! TEX root = ../main.tex
\documentclass[main]{subfiles}

\begin{document}


%%
%%
%% section や subsection を必要に応じて用い,本文を記述する.
\section{はじめに }
本研究では,Webフロントエンド開発の環境構築課題を解決するための
Webアプリケーション「Teamenver」を開発・検証する.
近年のWebフロントエンド開発においては,ライブラリやツールが頻繁に登場し,開発の際に適切に選択できないという問題が生じている.
また,選定後もライブラリ選定の意図が共有されないため,チームメンバーの納得を得られているかが不透明である.
それらは更なる提案やチームでのコミュニケーションの阻害に繋がる.
Teamenverは技術選択の透明性を提供し,選定意図の文書化を可能にする.
これにより,チーム内のコミュニケーションが改善され,初学者を含むメンバーでも適切な開発が可能となる.
その有効性を,Teamenverを使用した開発チームとそうでないチームの開発体験の比較により検証する.
\end{document}
