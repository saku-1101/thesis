%! TEX root = ../main.tex
\documentclass[main]{subfiles}

\begin{document}


%%
%%
%% section や subsection を必要に応じて用い,本文を記述する.
\section{概要 }
⭐️Figmaのデザインを貼り付けてそれを用いて説明しなければならない!
本論文では,モダンWebアプリケーションの環境構築における課題について考察し,それに対する改善を図るためのWebアプリケーションを提示する.現代のWebフロントエンド開発は急速に進化しており,新しいライブラリやツールが頻繁に登場しているため,現在のプロジェクトにとって最適な選択が難しい.

初期状態でのライブラリ選定においても,どのようなライブラリを導入すればよいかが明確でなく,適切な技術スタックを選択できないという問題が生じている.また,選定後もライブラリ選定の意図が共有されず,チームメンバーの納得を得られているかは不透明である.

さらに,適切なアーキテクチャ・レンダリングスタイルの選択が難しいことも懸念点として挙げられる.アーキテクチャやレンダリング・コンテンツデリバリーの選択がメンバーの習熟度によって変化する場合,非機能的な要求を満たせないアプリケーションになってしまう恐れがある.

さらに,メンバーの思考が追いにくいことも課題として挙げられる.ライブラリや技術選定の理由やメリット・デメリットが共有されないと,更なる提案やチームでのコミュニケーションが困難になり,他のメンバーが持つ知識や経験を活用したより良い開発体験の阻害が懸念される.

これらの課題を踏まえ,より良い環境構築を実現するためには,次のようなアプローチが考えられる.

1. ライブラリやツールの提案: モダンで信頼できるバージョンのライブラリやツールを提案し,プロジェクトにとって最適な選択が行えるようにサポートする必要がある.

2. 技術選定の透明性向上: ライブラリの選定理由や導入メリット・デメリットを文書化し,チーム内で共有することにより納得感を高めることが重要である.

3. 技術スタックの選定プロセスの明確化: 選択される技術スタックやアーキテクチャに関する意思決定プロセスを透明化し,関係者が意見を述べやすい環境を整備する必要がある.

4. チーム内の知識共有の促進: チームメンバー間で知識や経験を積極的に共有する文化を醸成し,他のメンバーが持つ知識に気づく機会を増やすことが大切である.

これらの改善策を実践することで,環境構築における課題を克服し,より生産性の高い開発環境を構築できる.結果として,アプリケーションの品質向上とユーザーエクスペリエンスの向上を実現できることが期待される.
\end{document}
