%! TEX root = ../main.tex
\documentclass[main]{subfiles}

\begin{document}


%%
%%
%% section や subsection を必要に応じて用い,本文を記述する.
\section{はじめに }
現代の Web フロントエンド開発の急速な進化により,ラ
イブラリやツールが頻繁に登場し,開発の際に適切に選択
できないという問題が生じている\cite{npm}\cite{medium.com}.
また,選定後もライブラリ選定の意図が共有されず,
チームメンバーの納得を得られているかが不透明なだけでなく,
更なる提案やチームでのコミュニケーションを阻害する.\\
 Teamenverは適切な技術選択とその透明性を提供し,選定意図の文書化を可能にする.\\
 本研究では,その有効性をTeamenverを使用した開発チームとそうでないチームの開発体験の比較により検証する.
\end{document}
